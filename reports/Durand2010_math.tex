\documentclass[]{article}
\usepackage{lmodern}
\usepackage{amssymb,amsmath}
\usepackage{ifxetex,ifluatex}
\usepackage{fixltx2e} % provides \textsubscript
\ifnum 0\ifxetex 1\fi\ifluatex 1\fi=0 % if pdftex
  \usepackage[T1]{fontenc}
  \usepackage[utf8]{inputenc}
\else % if luatex or xelatex
  \ifxetex
    \usepackage{mathspec}
  \else
    \usepackage{fontspec}
  \fi
  \defaultfontfeatures{Ligatures=TeX,Scale=MatchLowercase}
\fi
% use upquote if available, for straight quotes in verbatim environments
\IfFileExists{upquote.sty}{\usepackage{upquote}}{}
% use microtype if available
\IfFileExists{microtype.sty}{%
\usepackage{microtype}
\UseMicrotypeSet[protrusion]{basicmath} % disable protrusion for tt fonts
}{}
\usepackage[margin=1in]{geometry}
\usepackage{hyperref}
\hypersetup{unicode=true,
            pdfborder={0 0 0},
            breaklinks=true}
\urlstyle{same}  % don't use monospace font for urls
\usepackage{graphicx,grffile}
\makeatletter
\def\maxwidth{\ifdim\Gin@nat@width>\linewidth\linewidth\else\Gin@nat@width\fi}
\def\maxheight{\ifdim\Gin@nat@height>\textheight\textheight\else\Gin@nat@height\fi}
\makeatother
% Scale images if necessary, so that they will not overflow the page
% margins by default, and it is still possible to overwrite the defaults
% using explicit options in \includegraphics[width, height, ...]{}
\setkeys{Gin}{width=\maxwidth,height=\maxheight,keepaspectratio}
\IfFileExists{parskip.sty}{%
\usepackage{parskip}
}{% else
\setlength{\parindent}{0pt}
\setlength{\parskip}{6pt plus 2pt minus 1pt}
}
\setlength{\emergencystretch}{3em}  % prevent overfull lines
\providecommand{\tightlist}{%
  \setlength{\itemsep}{0pt}\setlength{\parskip}{0pt}}
\setcounter{secnumdepth}{0}
% Redefines (sub)paragraphs to behave more like sections
\ifx\paragraph\undefined\else
\let\oldparagraph\paragraph
\renewcommand{\paragraph}[1]{\oldparagraph{#1}\mbox{}}
\fi
\ifx\subparagraph\undefined\else
\let\oldsubparagraph\subparagraph
\renewcommand{\subparagraph}[1]{\oldsubparagraph{#1}\mbox{}}
\fi

%%% Use protect on footnotes to avoid problems with footnotes in titles
\let\rmarkdownfootnote\footnote%
\def\footnote{\protect\rmarkdownfootnote}

%%% Change title format to be more compact
\usepackage{titling}

% Create subtitle command for use in maketitle
\newcommand{\subtitle}[1]{
  \posttitle{
    \begin{center}\large#1\end{center}
    }
}

\setlength{\droptitle}{-2em}
  \title{}
  \pretitle{\vspace{\droptitle}}
  \posttitle{}
  \author{}
  \preauthor{}\postauthor{}
  \date{}
  \predate{}\postdate{}


\begin{document}

\section{Means of obtaining bathymetry from mass-conserved Manning's
equation}\label{means-of-obtaining-bathymetry-from-mass-conserved-mannings-equation}

\subsection{Intro}\label{intro}

Manning's McFli has an apparent contradiction. The system is such that
unknowns vary solely in time or space, whereas observations vary along
both time and space. Durand (2010) showed that the spatially varying
unknown can be estimated independently of the time-varying unknown, but
GaMo and I have shown that \(Q_t\) is unobtainable even if \(A_0\) is
known perfectly. What gives?!

The following is an attempt to elucidate the differences in these
approaches, and the conclusions reached by each one.

\subsubsection{Commonalities across
approaches}\label{commonalities-across-approaches}

Both approaches have the following aspects in common.

\begin{itemize}
\tightlist
\item
  Apply Manning's equation at multiple times and locations
\item
  Transform to linearize desired unknowns
\item
  Apply conservation of mass
\item
  Further manipulations to infer or show impossibility of inference
\end{itemize}

\subsubsection{Problem setup}\label{problem-setup}

Assume we have measurements of slope, width, and partial cross-section
area for T times and N locations within a mass-conserved river segment.
Then we can write

\[
\begin{aligned}
Q_{it} &= \frac{1}{n} W_{it}^{-2/3} A_{it}^{5/3}S_{it}^{1/2} \\
\implies Q^{3/5}_{it} & = A_{it}\Big(\frac{1}{n} W_{it}^{-2/3} S_{it}^{1/2} \Big)^{3/5}
\end{aligned}
\] For simplicity let
\(x_{it} = \Big(\frac{1}{n} W_{it}^{-2/3} S_{it}^{1/2} \Big)^{3/5}\), so
that \(Q^{3/5}_{it} = A_{it}x_{it}\).

\subsubsection{2. Linearize}\label{linearize}

The linearization to isolate \(Q\) is necessarily different from that to
isolate \(A_0\). This is at the heart of the difference in inferrability
of these two quantities.

From the above, we have \[
Q^{3/5}_{it} = A_{it}x_{it} = (A_{0,i} + \delta A_{it}) x_{it}
\]

Note that this equation is linear in both \(A_{0,i}\) and \(Q_{it}\).

\subsubsection{3. Mass conserve}\label{mass-conserve}

Steady-state mass conservation requires that \(Q_{it} = Q_{i't}\) for
all \(i, i'\) in \(1, 2, ..., N\). This is equivalent to\\
\[
\sum_{i = 1}^N\omega_i A_{it} x_{it} = 0
\] for any set of \(\omega_i\)'s such that
\(\sum_{i = 1}^N \omega_i = 0\). In matrix notation, this is

\[
(\mathbf{A} \circ \mathbf{X}) \mathbf{\omega} = \mathbf{0}
\] where \(\circ\) denotes the elementwise product. We can now split
this into the unobserved and observed parts of area: \[
((\mathbf{A_0} + \mathbf{\delta A}) \circ \mathbf{X}) \mathbf{\omega} = (\mathbf{A_0} \circ \mathbf{X}) \omega + (\mathbf{\delta A} \circ \mathbf{X}) \mathbf{\omega} =  \mathbf{0} \\
\implies (\mathbf{A_0} \circ \mathbf{X}) \omega = (-\delta \mathbf{A} \circ \mathbf{X}) \omega
\] A little math shows that the following representations of the LHS are
equivalent: \[
(\mathbf{A}_0 \circ \mathbf{X}) \omega = \mathbf{X} \mathbf{\Omega} \mathbf{a}_0 = \mathbf{X} (\mathbf{\omega} \circ \mathbf{a_0})
\] where \(\mathbf{\Omega}\) is the diagonal matrix whose diagonal
entries are the elements of \(\mathbf{\omega}\) and \(\mathbf{a_0}\) is
the the vector of \(A_0\)'s that is repeated to form the matrix
\(\mathbf{A_0}\), i.e.

\$\$ \mathbf{A_0} =

\begin{bmatrix} 

\mathbf{a_0} \\
\mathbf{a_0} \\
\vdots \\
\mathbf{a_0} \\

\end{bmatrix}

=

\begin{bmatrix} 

A_{0,1} & A_{0,2} & ... & A_{0,N} \\
A_{0,1} & A_{0,2} & ... & A_{0,N} \\
\vdots \\
A_{0,1} & A_{0,2} & ... & A_{0,N} \\


\end{bmatrix}

\$\$

\subsubsection{4. Inference procedure}\label{inference-procedure}

Given the above, the vector \(\mathbf{a_0}\) can be inferred by solving
the following linear model:

\[
\mathbf{X} \mathbf{\Omega} \mathbf{a}_0 = (-\delta \mathbf{A} \circ \mathbf{X}) \omega
\]

The model matrix, \(\mathbf{X \Omega}\), may conceivably have fewer rows
than columns; inference can be aided by selecting multiple \(\omega\)
vectors and appending the system as follows:

\$\$

\begin{bmatrix}
\mathbf{X} \mathbf{\Omega_1} \\
\mathbf{X} \mathbf{\Omega_2} \\
\vdots \\
\mathbf{X} \mathbf{\Omega_n}
\end{bmatrix}

\mathbf{a}\_0

=

\begin{bmatrix}
(-\delta \mathbf{A} \circ \mathbf{X}) \omega_1 \\
(-\delta \mathbf{A} \circ \mathbf{X}) \omega_2 \\
\vdots \\
(-\delta \mathbf{A} \circ \mathbf{X}) \omega_n

\end{bmatrix}

\$\$

\subsubsection{5. Application to Pepsi
datasets}\label{application-to-pepsi-datasets}

Some results on Pepsi cases.

Relative error by case:

\begin{figure}
\centering
\includegraphics{../graphs/A0_relerr1.png}
\caption{}
\end{figure}

Relative error with linear model \(R^2\):

\begin{figure}
\centering
\includegraphics{../graphs/A0_relerr_R2.png}
\caption{}
\end{figure}

Interestingly, the relative error on slope inference does not have a
strong correlation with model fit.

Here is the model fit for all cases (that didn't produce errors):

\begin{figure}
\centering
\includegraphics{../graphs/A0lm_fit.png}
\caption{}
\end{figure}

Some other points:

\begin{itemize}
\tightlist
\item
  Not related to mass conservation / steady-state assumptions
\item
  Manning's equation appears to fit relatively well in all cases.
\end{itemize}

\paragraph{Clues as to what is affecting
fit}\label{clues-as-to-what-is-affecting-fit}

Inspecting the relationships between variables in log space reveals some
nonlinear relationships (which would mean nonmultiplicative in linear
space). This appears to be a violation of Manning's equation
assumptions.

\begin{itemize}
\tightlist
\item
  Here's Po (poor linear model fit but good A0 inference):
\end{itemize}

\includegraphics{../graphs/Po_termplot_a.png}
\includegraphics{../graphs/Po_termplot_s.png}
\includegraphics{../graphs/Po_termplot_w.png}

\begin{itemize}
\tightlist
\item
  Here's Connecticut (good linear model fit but poor A0 inference):
\end{itemize}

\includegraphics{../graphs/Connecticut_termplot_a.png}
\includegraphics{../graphs/Connecticut_termplot_s.png}
\includegraphics{../graphs/Connecticut_termplot_w.png}


\end{document}
